\documentclass[a4paper]{article}
\usepackage[utf8x]{inputenc}

\usepackage{booktabs}
\usepackage{tabularx}
\usepackage{longtable}

\usepackage{hyperref}

\usepackage{tabularray}
\usepackage{graphicx}

\usepackage[table]{xcolor}
\usepackage[english]{babel}
\usepackage{csquotes}
\usepackage{url}
\usepackage[style=ieee,backend=biber]{biblatex} % Bibliography
\usepackage{isomath}
\usepackage{amsmath}
\usepackage{newtxmath}
\usepackage{listings}
\usepackage{float}
\usepackage{bm}
\usepackage{ stmaryrd }

\usepackage{geometry}
\geometry{
    a4paper,
    left=25mm,
    right=25mm,
    top=25mm,
    }
    
\title{ICASSP '25 notes and interesting posters}
\author{Benedikt Kantz}
\begin{document}

\maketitle

\section{Montag Vormittag}
\subsection{Tutorial: Generative AI and Model Optimization}
Problem: (compute) cost, current foundation models not sustainable
Solutions:
\subsubsection{Sparsity}
\begin{itemize}
    \item[$\rightarrow$] scalability, less overfitting, interpretability, adaptive ways to introduce sparsity
    \item post training: optimal brain damage (OBD)/ optimal brain surgery (OBS)
          \begin{itemize}
              \item dropout by contribution to error, scale by Hessian $\mathcal{H}$ contribution

          \end{itemize}
    \item training:
          \begin{itemize}
              \item L1-loss: Convex optim.; no free lunch: initial model very large!, more eqs.
              \item exaustive: very expensive
              \item greedy/evolutionary solutions: StOMP, GOMP based on L0-norm, but very effective
          \end{itemize}
    \item pre-training
          \begin{itemize}
              \item SET
              \item randomly initial init $\rightarrow$ evolutionary
          \end{itemize}
    \item architecutral: grow and shrink networks...
\end{itemize}
Problem: doesn't really work with LMs (empirical study), but well for other networks (esp. low-weight dropout)

\subsubsection{Compression}
\begin{itemize}
    \item filter: storage compresion
    \item low rank factorization ($\neq$ LoRA), during train time not fine-tuning
    \item knowledge distillation
\end{itemize}

\section{Dienstag Nachmittag}

\subsection{Talk: Underwater Communications}

\begin{itemize}
    \item Problem: very slow comm underwater, $\approx$10~kHz range
    \item Towards moving target, Doppler correction using active SP correction, very manual work
\end{itemize}
Comment: interesting manual process, tedious work to sample


\section{Mittwoch Nachmittag}

\subsection{Talk: AI+SP}
Comment: some basics on diffusion/transformers, a little bit of SP in NNs


\section{Donnerstag Vormittag}

\subsection{Talk: Multiomics}

\begin{itemize}
    \item Genomics: DNA understanding
    \item Transcriptomics: DNA->RNA understanding
    \item Proteomics: RNA->Protein structures
    \item Knowledge graphs: how do these systems influnce each other
    \item Flow:
          \begin{itemize}
              \item identify DNA mutation that triggers illness
              \item find possible RNA mechanism
              \item find good fitting small ring structure
              \item[$\rightarrow$] check for side effects in knowledge graph! (certain protein effects unwantend)
              \item[$\rightarrow$] then test  $\rightarrow$ animal tests, reduce through ML!
          \end{itemize}
    \item Graph diffusion for drug discovery: noise schedule for diffusion essential, i.e. cosine-square schedule
          \begin{itemize}
              \item diffuse graphs from atoms \& edges as adjacency matrix
              \item what is noise: discrete noise: each atom is discrete state $\implies$ graph structure undergoes state transition change
              \item naive: uniform structure, not really chemically sensible - conditional probabibilites $\implies$ not uniform but marginal distribution of molecules in training (just logical!), same for edge (with deletion!)
              \item one step further: consider carbon rings, restriction based on maximum bonds of atom (freie radikale)
              \item SMILE-file, QED: Quantitative Esitmate of Drug likeness (from RDKit)
              \item Existing methods: Time-consuming, progress slow, very few good molecules
              \item Their work: jointly perturb rings+nodes
              \item other approaches: motives as super-node with rings, difficulty: ring attachments - only $\approx$1 \% improvement!
              \item novelty however high, one molecule of them even patented!
          \end{itemize}
\item Knowledge graphs:
\begin{itemize}
    \item GNN link prediction
    \item none of the existing benchmarks include features!
    \item maybe talk to author!
\end{itemize}
\end{itemize}
Comment: focused on drug discovery using diffusion, not much on multiomics\dots
\section{Lectures/Orals}
\begin{longtblr}{|X|X|X|}  
\hline 
Lecture & URL &  Notes  \\  
\hline 
\textbf{Diversity{-}Seeking Techniques for Red{-}Teaming LLMs} & \url{https://ieeexplore.ieee.org/document/10890844/} & Add RL{-}Loss to train similar to GAN by backpropagating if the model returns very similar output (i.e. discrimimnator){-}> Very fragile learning;  Limited further studies\\ 
\hline 
\textbf{FDR Control for Complex{-}Valued Data} & \url{https://ieeexplore.ieee.org/document/10889705} & similar to LASSO; sparsifying system under certain guarantees\\ 
\hline 
\textbf{SpectralCam: High{-}Resolution Low{-}Cost Spectral Imaging Using DSLR Cameras} & \url{https://ieeexplore.ieee.org/document/10887725} & Interesting concept of applying photo filter to DSLR sensor, Bayesian pattern restoration "learned" using diffusion \& attenuation mtx\\ 
\hline 
\textbf{Fusing Multimodality of Large Language Models and Satellite Imagery} & \url{https://ieeexplore.ieee.org/document/10889624} & Could be interesting in combination with HEREDITARY geospatial data once we have access\\ 
\hline 
\textbf{Controllable Forgetting Mechanism for Few{-}Shot Class{-}Incremental Learning} & \url{https://arxiv.org/pdf/2501.15998} & Using embedding space to classify, add new classifier based on distance, seems rather hyperparameter{-}sensitive\\ 
\hline 
\end{longtblr}

\section{Posters}
\include{posters}

\end{document}