\documentclass[a4paper]{article}
\usepackage[utf8x]{inputenc}

\usepackage{booktabs}
\usepackage{tabularx}
\usepackage{longtable}

\usepackage{hyperref}

\usepackage{tabularray}
\usepackage{graphicx}

\usepackage[table]{xcolor}
\usepackage[english]{babel}
\usepackage{csquotes}
\usepackage[style=ieee,backend=biber]{biblatex} % 
\addbibresource{references.bib}
\usepackage{isomath}
\usepackage{amsmath}
\usepackage{newtxmath}
\usepackage{hyperref}
\usepackage{listings}
\usepackage{float}
\usepackage{bm}
\usepackage{ stmaryrd }
\usepackage{enumitem}
\setlist{
 itemsep=0.1em,
 parsep=0.1em
} % or \setlist{noitemsep} to leave space around whole list

\usepackage{geometry}
\geometry{
    a4paper,
    left=25mm,
    right=25mm,
    top=25mm,
    bottom=25mm,
    }
    
\title{IEEE VIS '25 notes and interesting posters}
\author{Benedikt Kantz}
\date{Nov 02-07 2025}
\begin{document}

\maketitle
\tableofcontents
\section{Workshshops}
\subsection{alt.VIS (Sunday Morning)}
\url{https://altvis.github.io/}
\subsubsection{Data Melodification FM}
\subsubsection{Visualization Was Here}
\subsubsection{The Fuzzy Front Ends}

Very interesting talks, Melodification could be part of survey for HCDA. TLX-on-TLX by Daniel was also a good throwback to the NASA-TLX, complete with a \LaTeX template to emulate the old typewriter-paper feel.
\ldots
\subsection{Bio+MedVIS (Sunday Morning): Redesign Challenge Introduction}
\begin{itemize}
    \item Spectral analysis of 31P-MRS output of 9 individuals
    \item Dashboards to show spectra/peaks with rations of metabolic ranges
    \item Difference between subjects focus of many works, ratios/difference
\end{itemize}
\subsection{SciVis (Sunday Afternoon)}
\begin{itemize}
    \item More focus on specific solutions
    \item Inverse problem analysis could be interesting?
    \item A lot of dimensionality reduction, used PCA which could be flawed in the current setting
    \item Filtering on parallel coordinates/coordinates was very popular too, could be interesting to replace by target function and weight to allow fuzzy optimisation and apply Pareto front for our approach.
\end{itemize}

\subsection{LDAV (Monday Morning)}
\subsubsection{Opening \& Keynote}
\begin{itemize}
    \item Keynote: Kwan-Liu Ma
    \item Extract data, integrate viz in flows, integration AI/ML
    \item INR:  implicit neural representations, for data reduction, i.e. encode dataset in model (related to PINNs, kind of as data/observations encoded in NN), new method: Gaussian Splatting, kind of related as NERF supplanted by Splatting
    \item Glyph-based viz, after spatial aggregation (using Voronoi patterns) \footnote{\url{https://arxiv.org/pdf/2506.23092v1}}
    \item Uncertainty:  2D uncertainty: use small multiples, limitations of direct interpretation (requires understanding) \footnote{\url{https://arxiv.org/pdf/2012.11109}}, ClimateSOM to encode distribution of uncertainty (instead of summary)
    \item
\end{itemize}
\subsubsection{Extracting Complex Topology from Multivariate Functional Approximation: Contours, Jacobi Sets, and Ridge-Valley Graphs} Guanqun Ma, David Lenz, Hanqi Guo, Tom Peterka, \textbf{Bei Wang}

\begin{itemize}
    \item Extract topology without having to sample from model?
    \item Reframe Ridge-Valley graph (complex topology) from contour graph
    \item Core approach: use gradient of INF to trace paths, then use these as splines, extract gradients from these, then get the Ridge-Valley graphs/Jacobi Graphs
    \item Very smooth lines (B-Splines\dots)
    \item Possible weakness: NN gradients non-smooth/well learned... (especially for higher dimensions!)
\end{itemize}
\subsubsection{Extremely Scalable Distributed Computation of Contour Trees via Pre-Simplification}
Mingzhe Li, Hamish Carr, Oliver Rübel, Bei Wang, Gunther H. Weber
\begin{itemize}
    \item
\end{itemize}
\subsubsection{ChatVis}
\begin{itemize}
    \item Use RAG to automate viz based on prompts, evaluated on 20 examples + agentic correction flow
    \item RAG is quite good\dots
\end{itemize}
\subsubsection{From Soup to Bricks: Fast Clustering of Fine-Grained AMR Hierarchies for Rendering on GPUs}
Stefan Zellmann, Ingo Wald
\begin{itemize}
    \item Optimize Kd-based rendering of flow/volumetric data.
\end{itemize}
\subsubsection{Lossy Parallel Visualization of Large-Scale Volume Data with Error-Bounded Image Compositing}
Yongfeng Qiu, Yuxiao Li, Xin Liang, Yafan Huang, Guanpeng Li, Sheng Di, Franck Cappello, Hanqi Guo
\begin{itemize}
    \item Compression for image exchange between nodes
    \item Use cross-combinations by using cross-exchange using  $\mathcal{O} (n \log n)$
    \item Calculate error based on alpha-blending/compression overlaying
    \item Can calculate upper bound for error.
\end{itemize}

\subsubsection{Managing Data for Scalable and Interactive Event Sequence Visualization}
Sayef Azad Sakin, Katherine E. Isaacs

\begin{itemize}
    \item use Kd-tree across event paths, would it be better to use binary trees to not mix non-spatially related tracks.
    \item SSIM as metric for visual similarity for viz
\end{itemize}

\subsection{VAST Challenge 2025}

\subsubsection{Welcome \& Introduction}
\begin{itemize}
    \item Hidden story extraction from dataset (fictional island of Oceanis)
    \item MC1:
    \item MC2: Bias detecion in a KG, conflict of Tourism/Fishing
    \item MC3: Last year illegal fishing, this year: secret tourism patter/atypical activities detection including deceptive/contradicting information
    \item DC: Design challenge, propose designs for diverse teams

\end{itemize}

\subsubsection{BAIT Dashboard (MC2)}
\begin{itemize}
    \item Really cool dashboard for bias/change detection
    \item Scale metaphor to analize impact
    \item \url{https://bait.ava25.dbvis.de/}
\end{itemize}

\subsubsection{Interactive Platform 4 VA of Suspicious patterns (MC3)}
\begin{itemize}
    \item Build KG from radio messages
    \item Discovering patterns/groups within networks, using topic modeling over messages
    \item[\ldots] and using similarity measures of content
\end{itemize}
\subsubsection{Intuitive Support for Query Construction (DC)}
\begin{itemize}
    \item Related works: \textbf{State of the Art
              in Multivariate Network Visualization}
    \item TreeMap-based viz of node types/edges \& hierarchies
\end{itemize}

\section{Conference}

\subsection{Opening  \& Keynote: Visualization as a Science/The path of Viz to Science}
\begin{itemize}
    \item Early Viz: ASCII plot to a row-based line printer, very simple yet effective, next line printer (vector graphics) of MC simulations of ray scattering in clouds
    \item Early start in Supercomputers, \enquote{Mental Images} on the first GPUs (create every imaginable image on the computer)
    \item Medical planning: volume data to finite element representation using segmentation and organ rendering (surgery planning)
    \item Feature extraction from flow modeling
    \item da Vinci Studio (before Cave Viz)
    \item His points: serious application, entire problem, identify core functionality, cross disciplinary;
    \item The Science of Visual Insight
          \begin{itemize}
              \item One dataset can tell different stories (e.g. log vs linear)
              \item Visualizations to make sense of data!
              \item The mental representation/model/concepts/semantic networks/language/knowledge gets updated (Tobias's \enquote{aha!}, for language - viz metaphors)
              \item \enquote{modern} worldview: smybolic - nature - social orders, given truths - viz is according to him in the middle, cannot capture all truths... (this seems like the Venn diagram meme), map to different \enquote{ontologies} (not in KG sense but philosophical sense\dots)
              \item Floridi's infosphere / related viz to various philosophical arguments about truth and understandings, Foccaults power structures
              \item Critical Visualization \footnote{\url{https://academic.oup.com/cybersecurity/article/1/1/93/2366512?login=false}}
              \item What makes a science a science?
                    \begin{enumerate}
                        \item Epistemic clarity: measure/truths can be validated
                        \item Theoretical articulation: ontology of visual objects
                        \item Normative guardrails: ethics of epistemic-ness, power, fairness?
                    \end{enumerate}
              \item Visual Knowledge - definition? - transparent in provenance, reusable, uncertainty
              \item Visualizing Arguments - every viz serves a claim! Make structure/provenance clear
              \item Predictive Understanding - toolset to verifiably/predict Understanding
              \item Power and Responsibility of Viz, Foucault's representations coupled with power.
              \item two major problems: how to gain and communicate true knowledge


          \end{itemize}

\end{itemize}
\subsubsection{Awards}
\begin{itemize}
    \item VGTC Lifetime award: Meister
    \item Best short paper: Toward a Logic of Generalization about Visualization as a Decision Aid
          \begin{itemize}
              \item How can viz be useful for decision making / epistemic framework: atomic viz parts vs. contextualized interpretation
              \item Formalize decision theory in viz (state - signal - interpretation - action)
          \end{itemize}
\end{itemize}

\subsection{Best Paper Awards}
\subsubsection{"They Aren't Built For Me": An Exploratory Study of Strategies for Measurement of Graphical Primitives in Tactile Graphics}
Areen Khalaila, Lane Harrison, Nam Wook Kim, Dylan Cashman
\begin{itemize}
    \item Accessibility for Data using Physicalisation
    \item Tasks for BLV are exponentially harder than for sighted people
    \item Evaluate errors of blind people on chart reading - some measures are similar to sighted people, some are very different (i.e. area measurements)
    \item Expected outcomes: inclusive design, remove tactile distraction
\end{itemize}
\subsection{ReVISit 2: A Full Experiment Life Cycle User Study Framework}
Zach Cutler, Jack Wilburn, Hilson Shrestha, Yiren Ding, Brian Bollen, Khandaker Abrar Nadib, Tingying He, Andrew McNutt, Lane Harrison, Alexander Lex
\begin{itemize}
    \item Really great tool!
\end{itemize}

\subsubsection{Beyond Problem Solving: Framing and Problem–Solution Co-Evolution in Data Visualization Design }
Parsons, Prakash Chandra Shukla

\subsubsection{Causality-based Visual Analytics of Sentiment Contagion in Social Media Topics}
\begin{itemize}
    \item Interesting causality maps (by assuming time progression is not just correlation\dots)
\end{itemize}
\subsection{Test-of-Time award}

\subsubsection{Vector field visualization}
\begin{itemize}
    \item Use templates for critical points (fingerprints?)
\end{itemize}
\subsubsection{Uncertainty Viz}
\begin{itemize}
    \item Use bands to visualize uncertainties (see master thesis?)
\end{itemize}
\subsubsection{Compression algorithm for 4D vector compression}
\begin{itemize}
    \item Compression on block-level, with simple Mutiply-Add
    \item Widely used, Open-Source (funding for porting!)
\end{itemize}

\subsubsection{Graph-Theoretic Scagnostics}
\subsubsection{Voyager}
Well-known viz browsers
\subsubsection{Reducing Snapshots to Points}
\begin{itemize}
    \item Analyze dynamic network by flattening them and performing dimensionality reduction and drawing them as time curves.
    \item Kind of similar to GNN (according to author)
\end{itemize}

\subsection{Maps \& Spatial Vis}

\subsubsection{"Mapping What I Feel": Understanding Affective Geovisualization Design Through the Lens of People-Place Relationships}

\begin{itemize}
    \item Emotional Mapping/Geography
    \item Identifying Emotional Visualization through survey (remove image/historical viz) - Affective Visualization
    \item Situated and personalized compared to other emotional viz.
    \item For Jes\'u : social/humanistic viz.
    \item
\end{itemize}

\subsubsection{Unveiling the Visual Rhetoric of Persuasive Cartography: A Case Study of the Design of Octopus Maps}
\begin{itemize}
    \item Octopus: literally?, historically famous (they creep into the data, rhetorically speaking)
    \item Survey of maps containing octopus with deep coding of maps, propaganda/misinformation maps deeply connected!
\end{itemize}

\begin{figure}[H]
    \centering
    \includegraphics[width=0.8\linewidth]{figures/octopus_map.png}
    \caption{From \textcite{OctuposMap2025}}
\end{figure}

\subsubsection{How do Data Journalists Design Maps to Tell Stories?}
\begin{itemize}
    \item Explore design space of these specifc maps
    \item Article sourrounding maps, accompying viz/supporting media
    \item Interactivity/Overlays\dots
    \item Interactive more challenging, less common (expensive!)
    \item Tight deadlines force less map inclusion, AI enables more inclusion (but at what cost)?
    \item A lot of users require locator maps, data literacy often a problem/readers often lost, require well-versed audience
    \item Limitations: biased towards Brazilian journalists, no mobile maps
\end{itemize}

\begin{figure}[H]
    \centering
    \includegraphics[width=0.8\linewidth]{figures/journalism_taxonomy.png}
    \caption{From \textcite{JournalisMap2025}}
\end{figure}

\subsubsection{Algorithmically-Assisted Schematic Transit Map Design: A System and Algorithmic Core for Fast Layout Iteration}
\begin{itemize}
    \item Their major contribution: interactive design, compared to Hannah Bast's greedy construction!
    \item Fast/Global algorithms survey
    \item Kind of limited user study (but only a few people)
\end{itemize}

\subsubsection{Algorithms for Consistent Dynamic Labeling of Maps With a Time-Slider Interface}
\begin{itemize}
    \item Time-Based labeling of images with time/geospatial info
    \item Use maximal information metric for labeling
    \item Query by adjusted slider
    \item Optimal + Valid Activity diagram - define constriants to display optimal viz! Using ILP for optimal labeling
    \item @Julian: could be interesting to display your optimal display of progressive viz on time series
\end{itemize}
\subsubsection{Volume-Based Space-Time Cube for Large-Scale Continuous Spatial Time Series}
\begin{itemize}
    \item Space-Time Cube/linked views/overlay for displaying time-based data on maps
    \item Does not work on large-scale data!
    \item Slicing of volume (effectively volume atop map)
\end{itemize}
\subsection{Vis 4 Science}
\subsubsection{Paraview-MCP: An Autonomous Visualization Agent with Direct Tool Use}

\begin{itemize}
    \item AI agents for Para View
    \item Reduce Barrier of Entry
    \item No user study, but present strong use case with Blender MCP
\end{itemize}

\subsubsection{Uncertain Mode Surfaces in 3D Symmetric Second-Order Tensor Field Ensembles}

\begin{itemize}
    \item Prior work: Topological Tensor field
    \item Tensor Mode $\rightarrow$ can be extracted to get contours/limits/extremes
    \item Uncertainty in Tensor Ensemble: Calculate Mean - then Mode (removes information), therefore - mode of each field, then calculate variance/\dots over these, viz these
    \item Current limitations: only for 1D-lines, current work: add more extension
\end{itemize}
\subsubsection{Virtual Ray Sampling for Direct Volume Rendering using Hermite Interpolation}
\subsubsection{SEG-RobustEye: Understanding medical image segmentation models}
\subsubsection{Scope Meets Screen: Lessons Learned in Designing Composite Visualizations for Marksmanship Training Across Skill Levels}
\subsubsection{Interactive Visual Analytics of Carbon Cycle Science}
\begin{itemize}
    \item Usually: box-flow diagrams (boxes represent storage, arrows movement)
    \item
\end{itemize}
\subsubsection{Analyzing Time-Varying Scalar Fields using Piecewise-Linear Morse-Cerf Theory}
\subsection{Abstract Physical Spaces}
\subsubsection{Don't Stop Me Now: Visualizing Disruptions in Railroad Networks}
\begin{itemize}
    \item Goal: exploring train disruption dataset
    \item Disrupted segments/temporal heatmap
    \item Use Case: Spreading of weather related delays and repair \enquote{cooldown}
    \item \url{https://trainviz.github.io/}
\end{itemize}

\subsubsection{Posterity: Balancing historical context and visual dynamism while visualizing a collection of American labor posters}
\begin{itemize}
    \item Nice demo with dimensionality reduction/embedding similarity to custom posters/clustering
\end{itemize}
\subsubsection{Chronotome: Real-Time Topic Modeling for Streaming Embedding Spaces}
\begin{itemize}
    \item Progressive Viz of Clusters
    \item Could be interesting for the RIS time viz of topics in laws?
    \item Cluster evolution over time in 3D!, time constant cluster by doing 2D clustering first then expanding to 3D
    \item Applied to 3 different models, and has the progressive aspects which sets it apart\dots
\end{itemize}
\subsubsection{Data-Driven Compute Overlays for Interactive Geographic Simulation and Visualization}
\begin{itemize}
    \item Montain viz of avalanche risk
    \item Web-based compute for dynamic overlay using LoD system + webGPU
    \item \url{https://webigeo.alpinemaps.org/}
\end{itemize}
\subsubsection{Embedding Atlas: Low-Friction, Interactive Embedding Visualization}
\begin{itemize}
    \item Main selling point: easy setup/sane defaults/fast
    \item Uses webGPU for fast rendering
    \item Easy viz config, reducing friction
\end{itemize}
\subsubsection{CFTree: Exploring Paths Through Counterfactuals}
\begin{itemize}
    \item Uses DECE for Counterfactuals
    \item Parallel coordinates over nodes
\end{itemize}
\subsubsection{Visualizing Climate Model Outputs with CliMAScope}
\begin{itemize}
    \item Full-On Design study with experts
\end{itemize}
\subsubsection{MC-INR: Efficient Encoding of Multivariate Scientific Simulation Data using Meta-Learning and Clustered Implicit Neural Representations}
\begin{itemize}
    \item Usually/Related: INR for compression
    \item Their approach: not only use a single INR, globally, but use multiple INRs clustered by k-Means
\end{itemize}
\subsection{Graphs and Networks}
\subsubsection{Envisage: Towards Expressive Visual Graph Querying}

\subsection{Dimensionality Reduction and Parameter Space Analysis}

\subsubsection{ClimateSOM: A Visual Analysis Workflow for Climate Ensemble Datasets}
\begin{itemize}
    \item Partition SOM into clusters
    \item Study with experts (this seems to be the deciding factor for many accepted papers!)
\end{itemize}
\subsection{SEAL: Spatially-resolved Embedding Analysis with Linked Imaging Data}
\begin{itemize}
    \item Gigapixel size with 100+ channels, millions of cells!
    \item Bridge spatial image / embedding
    \item Cells extracted and rearranged in embedding view - hybrid embedding view, render cells progressively
    \item Linked view dashboard, with lasso tools to select embedding view and show in overview.
    \item Calculate SHAP features based on model learned on dimensionality reduction (similar to our SciVis submission - we just use SG and use it as guidance)
    \item Eval: Use Case Study with three experts in three domain
    \item \url{http://sealvis.org/} using anywidet
\end{itemize}
\subsubsection{A Critical Analysis of the Usage of Dimensionality Reduction in Four Domains}
\begin{itemize}
    \item Survey of DR usage in domains
\end{itemize}
\subsubsection{Interactive Visual Analysis of Spatial Sensitivities}

\subsubsection{RSVP for VPSA : A Meta Design Study on Rapid Suggestive Visualization Prototyping for Visual Parameter Space Analysis}
\subsection{IEEE VIS Reviewing — On a Path to Self-Destruction?}
\textit{Petra Isenberg, Gunther Weber, Narge Mahyar, Niklas Elmqvist, Han-Wei Shen, Michael Sedlmair, Melanie Tory, Helwig Hauser, Bei Wang, Tamara Munzner}
\begin{itemize}
    \item Reviews are on a bumpy path - LMs, virtual conference, good (?): student reviews
    \item What we \textins{should} value in reviews (related to psychology panel)
    \item Survey results
    \begin{itemize}
        \item Gatekeeping in Viz -- less novelty, subjective quality, career systems; efficiency is going down (hundreds of review cycles) -- beer garden theory (spend your time effectively)
        \item Intransparent discussion -- reviewers change score, but not underlying review; one person (critical one) leads discussion, 
        \item Perceived randomness -- because it is a noisy sampling process; be in a good mood for reviewing ;), one (very convinced) person could lead discussion; recommendation: noise - a flaw in the human judgement
        \item Unqualified reviewers -- student reviewers (no weigh in review scores - that makes it feel valued for me\dots), unqualified PCS reviewers (how did that happen?), do not just average; also look for the good in the work, do not expect everything
        \item Unnecessarily negative reviews -- review fatigue, harsh gatekeeping, self-check expertise - positive framing (e.g. lacks justification instead of missing parameters)
        \item Low range -- people do not use the full range, update review after being convinced!
    \end{itemize}
    \item Discussion
\end{itemize}
\subsection{From data to meaning}
\subsubsection{Stitching Meaning: Practices of Data Textile Creators}
\begin{itemize}
    \item Survey of Creators
    \item Looking at (meta-)structure, meaning, data, color, motivation
\end{itemize}
\subsubsection{Story Ribbons: Reimagining Storyline Visualizations with Large Language Models}
\begin{itemize}
    \item Structural timelines from text(-stories)
    \item Character arcs \dots
    \item Using LMs and correction loops, interactive explanations and provenance to the texts
    \item Added a lot more fine-grained characters and themes compared to other methods
\end{itemize}
\subsubsection{$F^2$Stories: A Modular Framework for Multi-Objective Optimization of Storylines with a Focus on Fairness}
\begin{itemize}
    \item Fairness in  \textit{Viz}?
    \item Tradeoffs between global and local fairness (minorities)
    \item Networks: MILP used instead of global layouting
    \item Optimize crossings in story lines
\end{itemize}
\subsection{Transportation, Buildings, and Urban Vis}
\subsubsection{StressDiffVis: Visual Analytics for Multi-Model Stress Comparison}
\begin{itemize}
    \item Alternative Visualization for stress visualization for construction views
    
\end{itemize}
\subsubsection{StreetWeave: A Declarative Grammar for Street-Overlaid Visualization of Multivariate Data}
\begin{itemize}
    \item Maps, Urban analysis - Walkability score
    \item Design space survey of walkable/accesible neighbourhood street analysis.
\end{itemize}

\subsubsection{TraSculptor: Visual Analytics for Enhanced Decision-Making in Road Traffic Planning}
\begin{itemize}
    \item Model traffic demand on graph, explore options on graph
\end{itemize}
\section{Best Paper Awards according to me}
\begin{itemize}
    \item Smoothest Maps: Algorithmically-Assisted Schematic Transit Map Design: A System and Algorithmic Core for Fast Layout Iteration
    \item Longest Tentacles: Unveiling the Visual Rhetoric of Persuasive Cartography: A Case Study of the Design of Octopus Maps
    \item Fastest Avalanche: Data-Driven Compute Overlays for Interactive Geographic Simulation and Visualization
\end{itemize}
\section{Posters}
\include{posters}

\printbibliography

\end{document}